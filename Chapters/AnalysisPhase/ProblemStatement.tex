In a world where the need for critical embedded systems is rising, and where also the complexity of these systems is getting bigger and bigger, there was also a need for a way to bring the best performance possible in these systems. One of these solutions was a possibility to run multiple Operating Systems in the same hardware, thus improving the performance, which was achieved with the help of an Hypervisor.\\
\indent The Hypervisor provides a layer of abstraction between the hardware and the Operating Systems, making the connection to the hardware creating a transparent feel for the OS’s. It also manages the OS’s’ scheduling times. However, when the different OS’s running on top of the Hypervisor need to communicate between each other, the regular ways of communicating, through TCP/IP or other wireless based communication protocol, are not ideal and can slow-down the performance of the system, sometimes to an unusable state.\\
\indent For that reason, a standard Hypervisor will implement an Inter-Partition Communication, or IPC, mechanism, which allows seamless communication between the different partitions through the Hypervisor layer.\\
\indent However, for some critical systems, the communication needs to be as fast as possible and one solution is to accelerate it in hardware.
